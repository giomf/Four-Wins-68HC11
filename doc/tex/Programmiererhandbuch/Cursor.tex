\newpage
\section{Cursor}
    In diesem Kapitel wird der Cursor mit allen Funktionen und Bestandteilen erläutert.
    Darunter fallen unter anderem eine Variable (cursorColumn, deklariert in Viergewinnt.asm, Größe 1 Byte)
    zur Beschreibung der horizontalen-Position auf dem Board, welche die Spalte, die durch den Cursor ausgewählt wird repräsentiert und eine Konstante
    (cursorRow, in Viergewinnt.asm deklariert, hat den Wert 6) für die vertikale-Position des Cursors direkt unterhalb des Spielfelds.

    \subsection{Aufbau}
        Der Cursor wird auf dem LCD durch einen 6 Byte Breiten, ausgefüllten Pfeil unter dem Spielfeld (Zeile 6) dargestellt (Abbildung 7).

        \begin{figure}[H]
            \centering
            \includegraphics[scale=0.35]{img/cursor.png}    
            \caption{Cursor auf dem LCD}
        \end{figure}
    
    \subsection{Steuerung}
        Der Cursor startet nach betätigen des Resets (Taste 4 des Boards) oder bei Programmstart in der Mitte des Spielfeldes (Spalte 4).
        Er kann durch die Tasten 1 (nach links) und 3 (nach rechts) horizontal unter dem Spielfeld bewegt werden.
        Bei weiterer Bewegung und einer Cursorposition am Spielfeldrand erscheint der Cursor am gegenüberliegenden Spielfeldrand um schnelleres manövrieren zu ermöglichen (Abbildung 8).
        Bei Versersetzen des Cursors wird zuerst auf dem LCD der alte Cursor gelöscht,
        dann die Variable cursorColumn für links um 1 reduziert oder für rechts um 1 erhöht.
        Danach wird der Cursor erneut auf dem LCD an der geänderten cursorColumn angezeigt.

        \begin{figure}[H]
            \centering
            \includegraphics[scale=0.5]{img/board.png}    
            \caption{Board}
        \end{figure}